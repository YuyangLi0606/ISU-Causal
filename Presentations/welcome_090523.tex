\documentclass[handout]{beamer} %
\usetheme{CambridgeUS}
\usepackage[latin1]{inputenc}
\usefonttheme{professionalfonts}
\usepackage{times}
\usepackage{graphics}
\usepackage{tikz}
\usepackage{amsmath}
\usepackage{verbatim}
\usepackage{algorithm}
\usepackage{algorithmicx}
\usepackage{booktabs}

\usepackage[style=authoryear]{biblatex}

\addbibresource{~/Documents/grad_school/Research/references.bib}

% theorem
\BeforeBeginEnvironment{theorem}{
  \setbeamercolor{block title}{fg=red!80!black,bg=gray!10!white}
  \setbeamercolor{block body}{fg=black,bg=white}
}
\AfterEndEnvironment{theorem}{
    \setbeamercolor{block title}{fg=blocktitlefgsave,bg=blocktitlebgsave}
    \setbeamercolor{block body}{fg=blockbodyfgsave,bg=blockbodybgsave}
}


% definition
\BeforeBeginEnvironment{definition}{
  \setbeamercolor{block title}{fg=red!80!black,bg=gray!10!white}
  \setbeamercolor{block body}{fg=black, bg=white}
}
\AfterEndEnvironment{definition}{
    \setbeamercolor{block title}{fg=blocktitlefgsave,bg=blocktitlebgsave}
    \setbeamercolor{block body}{fg=blockbodyfgsave,bg=blockbodybgsave}
}

\renewcommand{\arraystretch}{1.5}

%\setbeamersize{text margin left=0.3mm,text margin right=0.1mm}
\DeclareMathOperator*{\argmin}{argmin}
\DeclareMathOperator*{\argmax}{argmax}
\DeclareMathOperator*{\expit}{expit}
\DeclareMathOperator*{\Cov}{Cov}
\DeclareMathOperator*{\Var}{Var}
\newcommand{\R}{\mathbb{R}}

\usepackage{hyperref}
\hypersetup{colorlinks=true, linkcolor=blue, filecolor=magenta, urlcolor=blue}

%\author[Azriel et al. (2022)]{Hyemin Yeon and Caleb Leedy}
\title[CIWG]{Causal Inference Working Group}

\begin{document}

\everymath{\displaystyle}
\setbeamertemplate{title page}[default][colsep=-4bp,rounded=true]
\setbeamertemplate{itemize items}[circle]
\setbeamercolor{block title}{bg=red!50,fg=black}
\frame{\titlepage}

\begin{frame}

\begin{center}
{\Large Welcome to the Causal Inference Working Group!}
\end{center}

\begin{itemize}
    \item<2> We hope that this will be a great opportunity to learn and 
    develop research in the field of causal inference.
\end{itemize}
    
\end{frame}

\begin{frame}{Goals}

    \begin{itemize}
        \item<2-> Understand the Causal Inference literature,
        \item<3-> Identify new areas of research, 
        \item<4-> Introduce new researchers to the Causal Inference field, and 
        \item<5-> Create a repository to facilitate future work.
    \end{itemize}
\end{frame}

\begin{frame}{Understanding the Literature}

    \begin{itemize}
        \item Most of what we are planning to do initially is to have people 
          present existing work.
        \item We hope that people will present papers, packages, and research 
          ideas.
        \item We want to understand what has been done \textit{and} especially
          what is missing.
    \end{itemize}
    
\end{frame}

\begin{frame}{Identifying New Areas of Research}

    \begin{itemize}
        \item When we know what needs to be done, we can fill this research gap. 
    \end{itemize}
    
\end{frame}

\begin{frame}{Introducing New Researchers to the Field}

    \begin{itemize}
        \item Many people here may be new to causal inference and we want to welcome
        you and get you started doing good research.
        \item Encourage good teaching practices when presenting.
    \end{itemize}
    
\end{frame}

\begin{frame}{Creating a Repository}

    \begin{itemize}
        \item We have a \href{https://github.com/calebleedy/ISU-Causal}{GitHub repo}. 
        \item The purpose of this is to make it easier for people to understand
        what already has been done, compare new ideas with existing methods, and
        provide a space for people to work together.
    \end{itemize}
\end{frame}

\begin{frame}{General structure}

Format 1: One main presentation
    \begin{itemize}
        \item 10 min: warm-up discussion before presentation 
        \item 30-40 min: main presentation
        \item 10 min: main discussion after presentation
    \end{itemize}

\vspace{0.3cm}

Format 2: Two related presentations
\begin{itemize}
    \item 5 min: warm-up discussion before presentation
    \item 15-20 min: main presentation
    \item 5 min: main discussion after presentation
\end{itemize}
\end{frame}


\begin{frame}{Expectations}

    \begin{itemize}
        \item<2-> Treat people with respect.
        \item<3-> Ask questions. (Even if you need to interrupt.)
        \item<4-> Give constructive feedback.
        \item<5-> Start and end on time.
    \end{itemize}
\end{frame}

\begin{frame}{Introductions}

\begin{itemize}
    \item<2-> Introduce yourself to a neighbor and
    \item<2-> Explain why you are interested in this causal inference group.
\end{itemize}

% Give 2 minutes. Then have everyone share with the group.
    
\end{frame}

\begin{frame}{Hopes}

\begin{itemize}
    \item Give opportunity for all to present existing and new ideas,
    \item Conduct good research,
    \item Learn from each other and give opportunities to learn from each other, and
    \item Work together to learn and produce work in an exciting and growing field.
\end{itemize}
    
\end{frame}

\begin{frame}{Ways to Help}

\begin{itemize}
    \item Present a paper
    \item Present a research idea
    \item Present an R (or Python) package and explain it with reproducible examples
    \item $\dots$
\end{itemize}

% While some people view presenting as a big deal, I would really encourage everyone to 
% be able to contribute, and I would prioritize more smaller presentations over 
% fewer smaller ones.
\end{frame}

\begin{frame}{Resources}

\begin{itemize}
    \item \href{https://github.com/calebleedy/ISU-Causal}{ISU Causal Inference GitHub repository}
    \item \href{https://sites.google.com/view/ocis/}{Online Causal Inference Seminar}
    \item \href{https://cran.r-project.org/web/views/CausalInference.html}{CRAN Task View: Causal Inference}
\end{itemize}
    
\end{frame}

% Survey:
% Name:
% Email:
% Do you want to be part of a group email for the Causal Inference Group? Y/N
% Why are you interested in this group?
% What causal inference research topics are you interested in?
% Which of these would you be willing to present?
% What papers would you recommend to be considered for presentation?
% What feedback do you have regarding the goals, expectations, or hopes of this group?

\end{document}
